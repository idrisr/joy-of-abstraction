\section{Set and Functions}
\subsection{Underlying importance}
Sets are the starting point for structures such as posets, monoids, topological
spaces and groups. Depending on the structure in question, the set is imbued
with some additional property---an ordering, a binary operation, or a notion of
closeness.
\begin{quote}
	The starting point for the maps between structures is the concept of a
	function---the basic notion of a map between sets.
\end{quote}
We then find a good notion of map between special structures by
looking at maps that preserve the special properties in question.
\begin{definition}
	Let $f, g$ be functions $A\rightarrow B$. We say $f = g$ as functions whenever
	$\forall a \in A, f(a) = g(a)$.
\end{definition}
\begin{ttta}
	How many different functions are possible between 3 inputs and 2 outputs? What
	if there were p inputs and k outputs?
	\begin{align*}
		2^3 &  & \text{3 inputs, 2 outputs} \\
		k^p &  & \text{p inputs, k outputs} \\
	\end{align*}
\end{ttta}
\begin{ttta}
	How many different functions are possible between $0$ inputs and $n$ outputs.
	How many different functions are possible between $n$ inputs and $0$ outputs.
	Can you interpret it?
	\begin{align*}
		n^0 & =1 &  & \text{input }\varnothing  \\
		0^n & =1 &  & \text{output }\varnothing \\
	\end{align*}
	The way to interpret this is that when $\varnothing$ is the output set, there is
	only one function from $n$ inputs.  Each input of $n$ is mapped to $\varnothing$
	like $n_i \mapsto \varnothing$. When the input is $\varnothing$ we get an absurd
	situation where one must produce an element of the $\varnothing$ to get an output.
	There is only way to produce such an absurdity.
\end{ttta}
\begin{ttta}
	\begin{enumerate}[label = {(\alph*)}]
		\item Show that sets and functions form a category.
		\item Check that the identity really behaves like an identity, both on
		      the left and on the right.
		\item Check that composition is
		      associative.\label{ttta:function-forms-category}
	\end{enumerate}
\end{ttta}
To form a category, there are two required properties: unitality and
associativity. Consider the category \textbf{Set} where sets are objects and
functions are arrows.
\begin{align*}
	(1_b \circ f)(x)                    & = 1_b(f(x))        &  &                       \\
	                                    & = f(x)             &  & \text{left identity}  \\
	(f \circ 1_a)(x)                    & = f(1_b(x))        &  &                       \\
	                                    & = f(x)             &  & \text{right identity} \\
	\left((f \circ g) \circ h\right)(x) & = (f \circ g)h(x)  &  & \text{}               \\
	                                    & = f(g(h(x)))       &  & \text{}               \\
	\left(f \circ (g \circ h)\right)(x) & = f((g\circ h)(x))                            \\
	                                    & = f(g(h(x)))       &  & \text{associativity}
\end{align*}
