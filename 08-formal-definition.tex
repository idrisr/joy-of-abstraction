\setcounter{section}{7}
\section{Categories, the Definition}
\begin{definition}
    The data in a category consists of objects and arrows.
    \begin{enumerate}
    \item Objects: a collection $\obc{}$ of objects
        \item Arrows: for objects $a, b\in \obc{}$, a collection
        $\mathcal{C}(a, b)$ of arrows $f: a\rightarrow b$. These arrows are not
        limited to functions, even though we use function notation to denote
        them;
    \end{enumerate}
\end{definition}
\begin{definition}
The structure in a category consists of identities and composition.
\begin{align*}
1_a : a \rightarrow a&&\text{identity}\\
f: a\rightarrow b&&\text{composition}\\
g: b\rightarrow c\\
g\circ f: a\rightarrow c
\end{align*}
\end{definition}
\begin{definition}
The properties in a category consists of unitality and associativity.
\begin{align*}
f: a\rightarrow b\\
f\circ 1_a = f = 1_b \circ f\\
    f:& a\rightarrow b\\
    g:& b\rightarrow c\\
    h:& c\rightarrow d\\
    (h\circ g)\circ f=&h\circ (g \circ f)
\end{align*}
\end{definition}
For a category to really be a category, there are seemingly four requirements:
identity arrows, composition, unitality, and associativity. But notice that the
unitality property requires identity arrows, and the associativity property
requires composition. So there are two properties, and both will fail if the
precondition is not met.
